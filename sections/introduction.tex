\documentclass[../notes.tex]{subfiles}

\begin{document}

The area of algebraic geometry originates with the study of solutions to polynomial equations
and systems of polynomial equations in multiple variables. This lecture begins with the more
classical approach to algebraic geometry and later transitions to the more modern approach.

\smallskip
\noindent
In the following lecture we denote with $k$ an algebraically closed field unless stated
otherwise. The classical view of algebraic geometry mainly deals with the following objects:

\smallskip

\begin{defi}{Algebraic sets}{}
  Let $k[T_1, \dots, T_n]$ be the set set of polynomials in variables $T_1, \dots, T_n$ with
  coefficients in $k$. For $M \subseteq k[T_1, \dots, T_n]$ we denote the zero locus
  of $M$ by
  \begin{align*}
    V(M) = \{x \in k^n \mid f(x) = 0 , \; \forall f \in M\}
  \end{align*}
  We call such subsets of $k^n$ algebraic sets.
\end{defi}

\smallskip
\noindent
In the case of $n = 1$ the algebraic sets are easy to determine.

\smallskip

\begin{ex}{}{}
  In the case of $n = 1$ the only algebraic sets of $k$ are finite sets and $k$ itself.
  For that we can distinguish the cases
  $(M) = (0)$ and $(M) \neq (0)$.
  In the first case, it immediately follows that $M = \{0\}$. Hence we get
  \begin{align*}
    V(M) = \{x \in k \mid f(x) = 0, \; \forall f \in M\} = \{x \in k \mid 0 = 0\}
    = k
  \end{align*}
  In the second case, the zero locus can be written as the intersection of finite sets
  \begin{align*}
    V(M) = \{x \in k \mid f(x) = 0, \; \forall f \in M\} =
    \bigcap_{f \in M} \{x \in k \mid f(x) = 0 \} 
  \end{align*}
  and is hence also finite.
\end{ex}

\smallskip
\noindent
In this setting affine varieties will be algebraic sets with some geometric structure
and varieties in general will be glued from affine varieties. In the modern setting
the objects will be schemes. Affine schemes will be some generalization of affine varieties
with the idea of replacing the ring $k[T_1, \dots, T_n]$ with arbitrary rings.
Schemes in general will then be glued from affine schemes.

\smallskip
\noindent
The starting goal of the lecture will be establishing a correspondence between the
sets
\begin{align*}
  \{ \text{subsets of } k[T_1, \dots, T_n] \} \longleftrightarrow
  \{ \text{subsets of } k^n\},
\end{align*}
where the correspondence from left to right is given by the operator $V$ defined previously.
For the correspondence for right to left we define the following operator:

\smallskip
\begin{defi}{}{}
  Let $X \subseteq k^n$ the we define the corresponding ideal to be
  \begin{align*}
    I(X) = \{f \in k[T_1, \dots, T_n] \mid f(x) = 0, \; \forall x \in X\}.
  \end{align*}
\end{defi}

\smallskip
\noindent
It is easy to check that for all $X \subseteq k^n$ this definition does indeed yield an ideal.
Assume that $f,g \in I(X)$ and $x \in X$ then $(f+g)(x) = f(x) + g(x) = 0 + 0 = 0$. Dropping now
the assumption that $g \in I(X)$ and assuming $g \in k[T_1, \dots, T_n]$ instead we get that
$(g \cdot f)(x) = g(x) f(x) = g(x) \cdot 0 = 0$. Therefore $I(X)$ is an ideal in
$k[T_1, \dots, T_n]$.

\smallskip
\noindent
The goal at first will now be to understand properties of this correspondence and more
specifically to characterize all $M \subseteq k[T_1, \dots, T_n]$ for which the
algebraic set $V(M)$ is non-empty.

\end{document}

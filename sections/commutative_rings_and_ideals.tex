\documentclass[../notes.tex]{subfiles}

\begin{document}

In the following lecture we will always consider a ring to be commutative and to have a unit.

\smallskip
\begin{defi}{}{}
  Let $R$ be a ring. A subset $I \subseteq R$ is called an ideal if $I \subseteq R$ is a subgroup
  with respect to addition and $a \in R$ and $b \in I$ implies $ab \in I$.
\end{defi}

\smallskip
\noindent
Assume just the first property for $I \subseteq R$. So let $I \subseteq R$ be a subgroup
with respect to addition. So we get a homomorphism of abelian groups
\begin{align*}
  \pi : R & \longrightarrow \faktor{R}{I} \\
  a & \longmapsto a + I
\end{align*}
and the following lemma:

\smallskip
\begin{lemm}{}{}
  The abelian group $R / I$ carries a ring structure such that $\pi$ is a ring homomorphism
  if and only if $I$ is an ideal.
\end{lemm}

\begin{proof}
  Assume $R / I$ carries a well defined multiplication such that $\pi$ is a ring homomorphism.
  Let now $a \in R$ and $b \in I$. So it follows that
  \begin{align*}
    \pi(ab) = \pi(a) \pi(b) =(a + I) \cdot (b + I) = (a + I) \cdot (0 + I) = 0 + I
  \end{align*}
  and on the other hand $\pi(ab) = ab + I$. Hence $ab \in I$ and $I$ is an ideal.

  \noindent
  Conversely if $I$ is an ideal we can define the multiplication structure on $R/I$
  by $(x + I)(y + I) = xy + I$. It is easy to check that this definition does not depend
  on the chosen representatives $x,y \in R$ and that $\pi$ becomes a ring homomorphism.
  
\end{proof}

\smallskip
\begin{ex}{}{}
  For $n \in \bN$ the ideal $n \bZ \subset \bZ$ results in the quotient ring $\bZ / n \bZ$.

  \smallskip
  \noindent
  A common quotient ring we will encounter is given by a subset $M \subset k[T_1, \dots, T_n]$
  and the ideal $(M)$. So the smallest ideal in $k[T_1, \dots, T_n]$, which contains $M$.
  The ring is then $k[T_1, \dots, T_n] / (M)$.
\end{ex}

\smallskip
\noindent
It turns out that this quotient construction satisfies the following universal property:

\smallskip
\begin{lemm}{Universal property of the quotient ring}{}
  Let $R \xrightarrow{\varphi} S$ be a ring homomorphism and $I \subseteq R$ be an ideal.
  If $\varphi(I) = \{0\}$ then there exists a unique ring homomorphism
  $\overline{\varphi}: R/I \to S$ such that the following diagram commutes:
  \begin{center}
    \begin{tikzcd}
      R \arrow[rr, "\varphi"] \arrow[dr, "\pi"'] & & S \\
      & \faktor{R}{I} \arrow[ru, "\exists ! \; \overline{\varphi}"', dashed] &
    \end{tikzcd}
  \end{center}
\end{lemm}

\begin{proof}
  We define the morphism $\overline{\varphi}$ by $\overline{\varphi}(a + I) = \varphi(a)$.
  First it should be mentioned that if $a,b \in R$ with $a - b \in I$ then
  \begin{align*}
    \overline{\varphi}(b + I) = \varphi(b) + \varphi(a - b) = \varphi(a) =
    \overline{\varphi}(a + I)
  \end{align*}
  and the definition $\overline{\varphi}$ does not depend on representatives. Second it should
  be checked that $\overline{\varphi}$ is indeed a ring homomorphism. So we compute
  for $a,b \in R$ that
  \begin{align*}
    \overline{\varphi}((a + I) + (b + I)) = \overline{\varphi}(a + b + I) = \varphi(a + b)
    = \varphi(a) + \varphi(b) = \overline{\varphi}(a + I) + \overline{\varphi}(b + I).
  \end{align*}
  The computation for multiplication is analogous. Lastly it is easy to see that
  $\overline{\varphi}(1 + I) = \varphi(1) = 1$ and so $\overline{\varphi}$ is a ring
  homomorphism.
  The fact that $\varphi = \overline{\varphi} \circ \pi$ immediately follows from the
  definition.

  \smallskip
  \noindent
  It only remains to check that $\overline{\varphi}$ is unique with this property.
  Let $\tilde{\varphi}: R/I \to S$ be another ring homomorphism such that
  $\varphi = \tilde{\varphi} \circ \pi$. Then it follows that
  \begin{align*}
    \tilde{\varphi}(a + I) = \varphi(a) = \overline{\varphi}(a + I)
  \end{align*}
  and hence that $\overline{\varphi}$ is unique.
  
\end{proof}

\subsection{Noetherian rings}

\begin{defi}{}{}
  A ring $R$ is called noetherian if every ideal $I \subseteq R$ is finitely generated.
  That is for every ideal $I \subseteq R$ there exists a finite subset $M \subseteq R$
  such that $I$ is the smallest ideal which contains $M$. We denote this with $I = (M)$.
\end{defi}

\smallskip
\noindent
This definition is characterized by an eqivalent property stated in the following lemma.

\smallskip
\begin{lemm}{Ascending chain condition}{}
  A ring $R$ is noetherian if and only if for every ascending chain
  \begin{align*}
    I_1 \subseteq I_2 \subseteq I_3 \subseteq I_4 \subseteq \cdots
  \end{align*}
  of ideals $I_j \subseteq R$ there exists an index $n \in \bN$ such that for
  all $m \geq n$ we have $I_m = I_n$.
\end{lemm}

\begin{proof}
  Assume that $R$ is not noetherian and let $I \subseteq R$ be an ideal that is not finitely
  generated. Then for every finitely generated ideal $I' \subseteq I$ there exists an
  element $f \in I \setminus I'$. We construct an ascending chain of ideals by defining
  $I_0 \coloneqq \{0\}$ and for every $n \in \bN$ choosing an elemnt $f_n \in I \setminus I_n$
  and setting $I_{n+1} \coloneqq (I_n \cup \{f_n\})$. By definition this is an ascending
  chain of ideals that at no step has an equality.

  \smallskip
  \noindent
  Assume now that $R$ is noetherian and consider an ascending chain of ideals
  \begin{align*}
    I_1 \subseteq I_2 \subseteq I_3 \subseteq I_4 \subseteq \cdots.
  \end{align*}
  Then the set $I = \bigcup_{n \in \bN} I_n$ is an ideal. For $a,b \in I$ there exists an
  $n \in \bN$ such that $a,b \in I_n$ and hence $a + b \in I_n$ which implies $a + b \in I_n$.
  Similarly if $a \in R$ and $b \in I$ then there is an $n \in \bN$ such that $b \in I_n$
  and thus $ab \in I_n$ which implies $ab \in I$.

  \noindent
  Due to $R$ being noetherian we get that $I$ is finitely generated. Let
  $\{f_1, \dots, f_n\} \subset I$ be a finite generating set. We then define
  \begin{align*}.
    m \coloneqq \max \{l \in \bN \mid \exists i \in \{1, \dots, n\}: \; f_i \notin I_l\}.
  \end{align*}
  Since for every $f_i$ there exists an index $l \in \bN$ such that $f_i \in I_l$ this subset
  of $\bN$ is bounded from above and hence contains a largest element.
  Now from $I_{m+1} \subseteq I$ and $\{f_1, \dots, f_n\} \subseteq I_{m+1}$ follows
  $I_{m+1} = I$ and hence
  \begin{align*}
    I_{m+1} = I_{m+2} = I_{m+3} = I_{m+4} = \cdots.
  \end{align*}
  
\end{proof}

\smallskip
\noindent
There are some ways to easily construct noetherian rings out of known noetherian rings.
The following lemma gives one such way.

\smallskip
\begin{lemm}{}{}
  If $R \xrightarrowdbl[]{\pi} R'$ is a surjective ring homomorphism and $R$ is
  noetherian then $R'$ is noetherian.
\end{lemm}

\begin{proof}
  Let $I \subseteq R'$ be an ideal. Then $\pi^{-1}(I)$ is an ideal
  and finitely generated by assumption. Let $\{f_1, \dots, f_n\} \subseteq \pi^{-1}(I)$
  be a generating set of $\pi^{-1}(I)$. Then we claim that
  $I = (\pi(f_1), \dots, \pi(f_n))$. Clearly we have $I \supseteq (\pi(f_1), \dots, \pi(f_n))$,
  since $\pi(f_i) \in I$ is equivalent to $f_i \in \pi^{-1}(I)$. If $f' \in I$
  then there exists $f \in R$ with $\pi(f) = f'$ and so $f \in \pi^{-1}(I)$.
  Therefore $f$ can be written as $f = \sum_{i = 1}^n g_i f_i$ with some $g_i \in R$.
  It follows that $f' = \pi(f) = \sum_{i = 1} \pi(g_i) \pi(f_i) \in (\pi(f_1), \dots, \pi(f_n))$.

  \smallskip
  \noindent
  With the equality shown it follows that $I$ is finitely generated.
  
\end{proof}

\smallskip
\noindent
Another way of finding new noetherian rings is given by the following theorem.

\smallskip
\begin{theo}{Hilbert's basis theorem}{}
  If $R$ is noetherian then $R[T]$ is noetherian.
\end{theo}

\smallskip
\noindent
Before we prove the theorem we formulate a few conclusions from the theorem and the previous
lemma. Additionally we need a preliminary lemma in the proof of the theorem.

\smallskip
\begin{cor}{}{}
  If $R$ is noetherian and $n \in \bN$ then $R[T_1, \dots, T_n]$ is noetherian.
  If $I \subset R[T_1, \dots, T_n]$ is an ideal then $R[T_1, \dots, T_n]/I$ is noetherian.
\end{cor}

\begin{proof}
  The first part of the statement follows from the theorem and induction by $n$.
  The second part of the statement follows from the lemma and the first part of the
  statement.
\end{proof}

\begin{lemm}{}{}
  \label{fin_subgen}
  Let $R$ be a ring and the ideal $I \subseteq R$ be finitely generated. Let furthermore
  $M \subseteq I$ be a generating set of $I$. Then there exists a finite subset
  $M' \subseteq M$ that generates $I$
\end{lemm}

\begin{proof}
  Let $\{f_1, \dots, f_m\}$ be a finite generating set for $I$.
  Then for every $i \in \{1, \dots, n\}$ we can write
  \begin{align*}
    f_i = \sum_{g \in M} b_{ig} g
  \end{align*}
  while almost all $b_{ig}$ vanish. So the set
  \begin{align*}
    M' \coloneqq \{g \in M \mid \exists i \in \{1, \dots, n\}: \; b_{ig} \neq 0\}
  \end{align*}
  is finite and we claim that $M'$ generates $I$.
  The inclusion $(M') \subseteq I$ is clear. So assume that $h \in I$.
  Becuase the $f_i$ generate $I$ it follows that there exist $a_1, \dots, a_n \in R$ such that
  \begin{align*}
    h = \sum_{i = 1}^n a_i f_i = \sum_{i = 1}^n \sum_{g \in M'} a_i b_{ig} g
    =  \sum_{g \in M'} \left( \sum_{i = 1}^n a_i b_{ig} \right) g
  \end{align*}
  and hence $h \in (M')$
\end{proof}

\begin{proof}[Proof of Hilbert's basis theorem]
  Assume $R[T]$ is not noetherian and let $I \subseteq R[T]$ be an ideal which is not finitely
  generated. Choose a polynomial $f_1 \in I$ of minimal degree $n_1$. For every $i \in \bN$
  choose a polynomial $f_i \in I \setminus (f_1, \dots, f_{i-1})$ of minimal degree $n_i$.
  Note that for $i \leq j$ we get $n_i \leq n_j$. For all $i \in \bN$ we denote the leading
  coefficient of $f_i$ with $a_i$. Since $R$ is noetherian it follows that
  $(a_i \mid i \in \bN) \subseteq R$ is finitely generated. By Lemma 1.10
  there exists an $m \in \bN$ such that $(a_1, \dots, a_m) = (a_i \mid i \in \bN)$
  and in particular there are $b_1, \dots, b_m \in R$ such that
  \begin{align*}
    a_{m+1} = \sum_{i = 1}^m b_i a_i.
  \end{align*}
  With that we set
  \begin{align*}
    g \coloneqq f_{m+1} - \sum_{i = 1}^m b_i T^{n_{m+1} - n_i} f_i
  \end{align*}
  Then we get that the $n_{m+1}$-th coefficient of $g$ vanishes and hence that
  $g$ is of lower degree than $f_{m+1}$. At the same time it must be the case that
  $g \in I \setminus (f_1, \dots, f_m)$, because if $g \in (f_1, \dots, f_m)$
  then
  \begin{align*}
    f_{m+1} = g + \sum_{i = 1}^m b_i T^{n_{m+1} - n_i} f_i \in (f_1, \dots, f_m)
  \end{align*}
  and that is not the case by the choice of $f_{m+1}$. But now $g$ being of lower
  degree than $f_{m+1}$ and $g \in I \setminus (f_1, \dots, f_m)$ contradicts the
  minimality of $n_{m+1}$. So the assumption that $I$ is not finitely generated
  must have been wrong.

\end{proof}

\begin{rem}{}{}
  We just showed that $k[T_1, T_2]$ and $\bZ[T]$ are noetherian. But even though
  every ideal $I$ in these rings is finitely generated, we can choose
  $I$ to require arbitrarily large (finite) generating sets.
\end{rem}

\begin{proof}
  
\end{proof}

\subsection{Units}

\begin{defi}{}{}
  Let $R$ be a ring. We call the elements of the set
  \begin{align*}
    R^* = \{x \in R \mid \exists y \in R: \; xy = 1\}
  \end{align*}
  the units in $R$.
\end{defi}

\smallskip

\begin{rem}{}{}
  Given $x \in R*$ the corresponding $y \in R$ with $xy = 1$ is unique.
  We denote this element with $x^{-1}$.
\end{rem}

\begin{proof}
  Let $y,y' \in R$ with $xy = xy' = 1$. Then we have $y = (xy')y = (xy)y' = y'$.
\end{proof}

\smallskip

\begin{defi}{}{}
  Let $R$ be a ring. A subset $S \subseteq R$ is called multiplicative if $1 \in S$
  and for all $a,b \in S$ it follows that $ab \in S$.
\end{defi}

\smallskip
\noindent Consider a ring $R$ and a multiplicative subset $S \subseteq R$.
The aim is to construct the ring of fractions which have only elements of $R$ as
enumerators and elements of $S$ as denominatos.

\smallskip
\begin{lemm}{}{}
  Define the relation $\sim$ on $R \times S$ by
  \begin{align*}
    (x,s) \sim (y,t) \quad :\Leftrightarrow \quad \exists u \in S: \; xtu = ysu.
  \end{align*}
  Then $\sim$ is an equivalence relation whose equivalence classes we denote by
  $[(x,s)]_{\sim} = \frac{x}{s}$.
\end{lemm}

\begin{proof}
  Form the definition reflexivity and symmetry are clear. Transitivity requires a short
  calculation. Let $(x,s) \sim (y,t)$ and $(y,t) \sim (z,u)$. so there
  exist $a,b \in S$ with $xta = ysa$ and $yub = ztb$. It follows that
  \begin{align*}
    xu(tab) = xta(ub) = ysa(ub) = yub(sa) = ztb(sa) = zs(tab)
  \end{align*}
  and so with $tab \in S$ we get $(x,s) \sim (z,u)$.
\end{proof}

\begin{lemm}{}{}
  For a ring $R$ and a multiplicaive subset $S \subseteq R$ we denote with
  \begin{align*}
    S^{-1}R = \left\{\frac{x}{s} \mid x \in R, \; s \in S\right\}
  \end{align*}
  the set of equivalence classes from the previous lemma. On this set we define an
  addition and a multiplication by
  \begin{align*}
    \frac{x}{s} + \frac{y}{t} = \frac{xt + ys}{st}
  \end{align*}
  and
  \begin{align*}
    \frac{x}{s} \cdot \frac{y}{s} = \frac{xy}{st}
  \end{align*}
  respectively. Then $S^{-1}R$ together with thes operations is a ring which we
  call the localization of $R$ by $S$.
\end{lemm}

\begin{proof}
  It is clear that the addition an multiplication are commutative and associative. Furthermore
  it is easy to see that $\frac{0}{1}$ is a neutral element with respect to addition
  and $\frac{1}{1}$ is neutral with respect to multiplication.
  It remains to check thhat both operations are well-defined and that
  multiplicaton distributes over addition. Let $(x,s) \sim (x',s')$ and
  $(y,t) \sim (y', t')$. So there are $u,v \in S$ such that $xs'u = x'su$ and
  $yt'v = y'tv$. Then we get
  \begin{align*}
    \frac{x}{s} + \frac{y}{t} = \frac{xt + ys}{st} = \frac{xts't'uv + yss't'uv}{sts't'uv}
    = \frac{x'sutt'v + y'tvss'u}{sts't'uv} = \frac{x't' + y's'}{s't'}
    = \frac{x'}{s'} + \frac{y'}{t'}
  \end{align*}
  and similarly we compute
  \begin{align*}
    \frac{x}{s} \cdot \frac{y}{t} = \frac{xy}{st} = \frac{xs'uyt'v}{sts't'uv} =
    \frac{x'suy'tv}{sts't'uv} = \frac{x'y'}{s't'} = \frac{x'}{s'} \cdot \frac{y'}{t'}.
  \end{align*}
  To check the distributivity let $x,y,z \in R$ and $s,t,u \in S$. Then we get
  \begin{align*}
    \frac{x}{s} \left( \frac{y}{t} + \frac{z}{u} \right) =
    \frac{x(yu + zt)}{stu} = \frac{xyu + xzt}{stu} = \frac{xyus + xzts}{s^2tu}
    = \frac{xy}{st} + \frac{xz}{su} = \frac{x}{s} \frac{y}{t} + \frac{x}{s} \frac{z}{u}.
  \end{align*}
\end{proof}

\smallskip
\noindent
We get that the localization satisfies a universal property.

\smallskip
\begin{lemm}{Universal property of the localization}{}
  For a ring $R$ and a multiplicative subset $S \subseteq R$ consider the ring homomorphism
  \begin{align*}
    j: R & \longrightarrow S^{-1}R \\
    x & \longmapsto \frac{x}{1}
  \end{align*}
  If $\varphi : R \to R'$ is a ring homomorphism such that $\varphi(S) \subseteq (R')^*$
  then there exists a unique ring homomorphism $\overline{\varphi}: S^{-1}R \to R'$
  such that the diagram
  \begin{center}
    \begin{tikzcd}
      R \arrow[rr, "\varphi"] \arrow[dr, "j"'] & & R' \\
      & S^{-1}R \arrow[ru, "\exists ! \; \overline{\varphi}"', dashed] &
    \end{tikzcd}
  \end{center}
  commutes.
\end{lemm}

\begin{proof}
  First we construct the homomorphism $\overline{\varphi}$. Since for every
  $s \in S$ the image $\varphi(s)$ is invertible in $R'$ we define
  \begin{align*}
    \overline{\varphi} \left(\frac{x}{s} \right) = \varphi(x) \varphi(s)^{-1}.
  \end{align*}
  Then clearly
  \begin{align*}
    \varphi(x) = \varphi(x) \varphi(1)^{-1} = \overline{\varphi} \left(\frac{x}{1} \right)
    = \overline{\varphi}(j(x)).
  \end{align*}
  Futhermore it follows immediately form the fact that $\varphi$ is a ring homomorphism
  that $\overline\varphi$ is also one. It remains to check that $\overine\varphi$ is
  well-defined. Let $(x,s) \sim (x',s')$ by the element $u \in S$. Then we get
  \begin{align*}
    \overline{\varphi} \left(\frac{x}{s} \right) = \varphi(x) \varphi(s)^{-1}
    = \varphi(xs'u) \varphi(ss'u)^{-1} = \varphi(x'su) \varphi(ss'u)^{-1} = \varphi(x')
    \varphi(s')^{-1} = \overline{\varphi} \left(\frac{x'}{s'} \right).
  \end{align*}
  Let $\tilde\varphi$ be another homomorphism that makes the diagram commute.
  Then it follows that
  \begin{align*}
    \overline{\varphi} \left(\frac{x}{s} \right) = \varphi(x) \varphi(s)^{-1}
    = \tilde{\varphi} \left(\frac{x}{1} \right) \tilde{\varphi} \left(\frac{s}{1} \right)^{-1}
    = \tilde{\varphi} \left(\frac{x}{1} \right) \tilde{\varphi} \left(\frac{1}{s} \right)
    = \tilde{\varphi} \left(\frac{x}{s} \right).
  \end{align*}
\end{proof}

\smallskip
\noindent
Similarly to the quotient ring the localization also preserves the property of being
noetherian.

\begin{lemm}{}{}
  If $R$ is a noetherian ring and $S \subseteq R$ is a multiplicative subset then
  $S^{-1}R$ is noetherian.
\end{lemm}

\begin{proof}
  From Hilbert's basis theorem it follows that $R[T]$ is noetherian. Considering the surjective
  homomorphism $R[T] \to S^{-1}R$, which sends constants $x \in R[T]$ to $x$ and sends
  $T$ to $\frac{1}{s}$ we get that $S^{-1}R$ is noetherian.
\end{proof}

\subsection{Prime ideals and maximal ideals}

\begin{defi}{}{}
  Let $R$ be a ring. An ideal $I  \subseteq R$ is called a prime ideal if the quotient ring
  $R/I$ is a domain.
\end{defi}

\smallskip
\noindent
We have the following equivalent characterization of prime ideals:

\smallskip

\begin{lemm}{}{}
  Let $R$ be a ring. An ideal $I \subseteq R$ is a prime ideal if and only if for all
  $a,b \in R$ with $ab \in I$ it follows $a \in I$ or $b \in I$.
\end{lemm}

\begin{proof}
  Let first $R/I$ be a domain and $a,b \in R$ be such that $ab \in I$.
  Then it follows that in the quotient $R/I$ we have $(a+I)(b+I) = ab + I = 0 + I$ and
  because $R/I$ is a domain it follows that $a + I = 0 + I$ or $b + I = 0 + I$.
  Hence $a \in I$ or $b \in I$.

  \smallskip
  \noindent
  Conversely assume that $I \subseteq R$ is a prime ideal and let
  $(a + I)(b + I) = 0 + I$. So it follows that $ab \in I$ and because $I$ is prime
  we have $a \in I$ or $b \in I$. So $a + I = 0 + I$ or $b + I = 0 + I$ making
  $R/I$ a domain.
  
\end{proof}

\smallskip

\begin{defi}{}{}
  Let $R$ be a ring. An ideal $I \subseteq R$ is called maximal if $I \neq R$ and for
  all ideals $J \subseteq R$ with $I \subseteq J$ it follows that $J = I$ or $J = R$.
\end{defi}

\begin{rem}{}{}
  For any ideal $I \subseteq R$ the projection homomorphism
  \begin{align*}
    \pi: R & \longrightarrow \faktor{R}{I} \\
    x & \longmapsto x + I
  \end{align*}
  induces a bijection between the set of ideals $J \subseteq R$ with $I \subseteq J$ and
  the set of ideals of the quotient ring $R/I$.
  This happens by mapping an ideal $\overline{J} \subseteq R/I$ to $\pi^{-1}(\overline{J})$.
  It is easy to check that this is a homomorphism and that the inverse map is given by
  mapping an ideal $I \subseteq J \subseteq R$ to $\pi(J)$, which is an ideal because
  the projection is surjective.
\end{rem}

\smallskip
\noindent
This remark becomes useful in formulating the following
important characterization of maximal ideals.

\smallskip
\begin{lemm}{}{}
  Let $R$ be a ring and $I \subseteq R$ be an ideal. Then $I$ is maximal if and only if
  the quotient ring $R/I$ is a field.
\end{lemm}

\begin{proof}
  Assume that $I$ is maximal. This is the case if and only if
  there only exist two ideals $J \subseteq R$
  with $I \suseteq J$. By the previous remark this is equivalent to
  $R/I$ only having two ideals, which must be $(0),(1) \subseteq R/I$. This is the
  case if and only if $R/I$ is a field.
\end{proof}

\begin{theo}{}{}
  If $R$ is a ring and $R \neq 0$ then $R$ contains a maximal ideal.
\end{theo}

\begin{proof}
  Let $\cM$ be the set of all ideals $I \subseteq R$, which are not $R$ itself.
  Then we get that $\cM \neq \varnothing$ since $(0) \in \cM$. If $\cM' \subseteq \cM$
  is a totally ordered subset then the union $I' \coloneqq \bicup_{I \in \cM'} I$
  is an ideal and hence an upper bound. By Zorns Lemma it follows that $\cM$ contains
  a maximal ideal.
  
\end{proof}

\end{document}

\documentclass[../notes.tex]{subfiles}

\begin{document}
In the following chapter $k$ always denotes an algebraically closed field.

\smallskip

\begin{defi}{}{}
  Let $M \subseteq k[\uT] \coloneqq k[T_1, \dots, T_n]$. We define $V(M)$ to be
  the zero locus of all elements of $M$. That means
  \begin{align*}
    V(M) \coloneqq \{x \in k^n \mid f(x) = 0 \; \forall f \in M\}.
  \end{align*}
  For a subset $X \subseteq k^n$ we define
  \begin{align*}
    I(X) \coloneqq \{f \in k[\uT] \mid f(x) = 0 \; \forall x \in X\}.
  \end{align*}
  This gives us a correspondence between subsets of $k[\uT]$ and subsets of $k^n$.
\end{defi}

\smallskip
\noindent
Right away there are some statements that can be observed about these definitions.

\smallskip

\begin{rem}{}{}
  \begin{enumerate}[($(i)$)]
  \item The operators $V$ and $I$ are inclusion reversing. That means if
    $M \subseteq M' \subseteq k[\uT]$ then it follows that $V(M) \supseteq V(M')$
    and similarly if $X \subseteq X' \subseteq k^n$ then $I(X) \supseteq I(X')$.
    Furthermore for all $X \subseteq k^n$ it holds that $X \subset V(I(X))$.

  \item For all $X \subseteq k^n$ the set $I(X) \subseteq k[\uT]$ is an ideal.
  \end{enumerate}
\end{rem}

\begin{proof}
  $(i)$: Let $M \subseteq M' \subseteq k[\uT]$ and $X \subseteq X' \subseteq k^n$.
  Then for a point $x \in k^n$ then if $f(x) = 0$ for all $f \in M'$ then in particular
  $f(x) = 0$ for all $f \in M$. Similarly if some $f \in k[\uT]$ vanishes everywhere on $X'$
  then in particular it vanishes everywhere on $X$.

  \noindent Lastly if $x \in X$ and $f \in I(X)$ then $f(x) = 0$ by definition of $I(X)$
  and hence $X \subset V(I(X))$.

  \smallskip
  \noindent
  $(ii)$: Now let $f_1, f_2 \in I(X)$ and $x \in X$. Then it follows that
  $(f_1 + f_2)(x) = f_1(x) + f_2(x) = 0 + 0 = 0$. If now $g \in k[\uT]$ then
  $(gf_1)(x) = g(x)f_1(x) = g(x) \cdot 0 = 0$. Hence $I(X)$ is an ideal in $k[\uT]$.
  
\end{proof}

\noindent
With a similar proof to the second statement of the remark one can also show that
for a subset $M \subset k[\uT]$ we have $V(M) = V((M))$.

\smallskip
\begin{defi}{}{}
  A subset $X \subset k^n$ is called algebraic if $X = V(M)$ for some subset
  $M \subseteq k[\uT]$. We call an ideal $I \subseteq k[\uT]$ admissable
  if and only if $I = I(X)$ for some $X \subseteq k^n$.
\end{defi}

\smallskip
\noindent It should be remarked that for $X \subseteq k^n$ there exists $M \subseteq k[\uT]$
with $X = V(M)$ if and only if there exists an ideal $I \subseteq k[\uT]$
such that $X = V(I)$.

\smallskip
\begin{lemm}{}{}
  Let $\cA$ be the set of algebraic subsets of $k^n$ and $\cB$ be the set of
  admissable ideals of $k[\uT]$. Then the
  maps $I|_\cA: \cA \to \cB$ and $V|_\cB: \cB \to \cA$ are bijections and inverses of one another.
\end{lemm}

\begin{proof}
  Let $X$ be an algebraic subset. We always have $X \subseteq V(I(X))$.
  Since $X$ is algebraic there exists an ideal $J \subseteq k[\uT]$ such that $X = V(J)$.
  Then we claim that $J \subseteq I(V(J))$. If $f \in J$ and $x \in V(J)$ then
  $f(x) = 0$ by definition. It follows that $f \in I(V(J))$ and hence our claim.
  From this and the fact that $V$ is inclusion reversing we  conclude that
  $X = V(J) \supseteq V(I(V(J))) = V(I(X))$.
  
\end{proof}

\smallskip
\noindent
In the following chapter we seek to understand what algebraic sets and admissable ideals
are and we are looking for alternative ways to characterize them. The following statement
gives a bit more structure to the algebraic subsets of $k^n$.

\smallskip
\begin{lemm}{}{}
  The algebraic subsets of $k^n$ are the closed subsets of a topology, which we
  call the Zariski topology on $k^n$.
\end{lemm}

\begin{proof}
  It is easy to see that $V(\{1\}) = \varnothing$ and $V(\{0\}) = k^n$.
  Let $\{M_i\}_{i}$ be a collection of subsets of $k[\uT]$ and $M_1, M_2 \in k[\uT]$.
  Then we claim that $V(\bigcup_i M_i) = \bigcap_i V(M_i)$ and
  \begin{align*}
    V(M_1M_2) = V(\{f_1f_2 \mid f_1 \in M_1, \; f_2 \in M_2 \}) = V(M_1) \cup V(M_2).
  \end{align*}
  We first check that the first statement holds. If $x \in V(\bigcup_ i M_i)$
  then if $f \in M_i$ for some $i$ it follows that $f(x) = 0$. Hence $x \in V(M_i)$,
  where $i$ was chosen arbitrarily. So we get $V(\bigcup_i M_i) \subseteq \bigcap_i V(M_i)$.
  On the other hand if $x \in V(M_i)$ for all $i$ then $f(x) = 0$ for all $f \in \bigcup_i M_i$.
  Therefore $V(\bigcup_i M_i) \supseteq \bigcap_i V(M_i)$.

  \noindent Now we check that the second statement holds. If $x \in V(M_1)$ and we have
  $f \in M_1$ and $g \in M_2$ respectively it follows that
  $(fg)(x) = f(x)g(x) = g(x) \cdot 0 = 0$. Hence $x \in V(M_11M_2)$.
  On the other hand if $x \notin V(M_1) \cup V(M_2)$ there must exist $f \in M_1$ and
  $g \in M_2$ such that $f(x) \neq 0$ and $g(x) \neq 0$. So $(fg)(x) = f(x) g(x) \neq 0$,
  since $k[\uT]$ is a domain. Therefore $x \notin V(M_1 M_2)$.
  
\end{proof}

\begin{ex}{}{}
  In an example from the introduction we have already seen that the Zariski topology on
  $k^1$ is the cofinite topology. Recall that we have shown the closed sets to be all finite
  sets and $k$ itself.
\end{ex}

\subsection{First properties of the Zariski topology}

We notice that the Zariski topology is (T1). That means that singletons $\{x\}, \; x \in k^n$
are closed sets. Given $x = (x_1, \dots, x_n)\in k^n$ we have
\begin{align*}
  V(T_1 - x_1, \dots, T_n - x_n) = \{x\}.
\end{align*}
We give the ideal corresponding to $\{x\}$ a special name.

\smallskip
\begin{defi}{}{}
  For a point $x \in k^n$ we write $\fm_x = I(\{x\})$.
\end{defi}

\smallskip
\noindent
We can immediately observe that for a point $x = (x_1, \dots, x_n) \in k^n$
$\fm_x$ is the kernel of the evaluation homomorphism
\begin{align*}
  \ev_x : k[T_1, \dots, T_n] & \longrightarrow k \\
  f & \longmapsto f(x) = f(x_1, \dots, x_n).
\end{align*}
With the help of this view on $\fm_x$ we can colclude the following lemma.

\smallskip
\begin{lemm}{}{}
  Let $x = (x_1, \dots, x_n) \in k^n$. Then $\fm_x = (T_1 - x_1, \dots, T_n - x_n)$.
\end{lemm}

\begin{proof}
  We denote the ideal $(T_1 - x_1, \dots, T_n - x_n)$ with $I$. We have already shown that
  $J \subseteq I(V(J))$ for all ideals $J \subseteq k[\uT]$. In particular $I \subseteq \fm_x$.
  
\end{proof}


\end{document}
